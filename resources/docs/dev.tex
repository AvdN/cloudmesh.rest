\subsection{How To}

Components are written as YAML markup in files in the
\verb+resources/samples+ directory.

For example:

% \VerbatimInput{../samples/profile.yml}

\subsubsection{Components}

Each resource should have a \verb+description+ entry to act as
documentation. The documentation should be formated as  reStructuredText.

For example:

\begin{verbatim}
foo:
  description: |
    title description
    =================

    Parameters
    ----------

    - bar: what bar is for
    - baz: what baz is for

  bar: 42
  baz: a string
\end{verbatim}


\subsubsection{Generating}

Run the \verb+elements+ command:

\begin{verbatim}
cms admin elements <directory> <out.json>
\end{verbatim}

where

\begin{itemize}
\item \verb+<directory>+: directory where the YAML definitions reside
\item \verb+<out.json>+: path to the combined definition
\end{itemize}


For example:

\begin{verbatim}
cms elements resources/samples all.json
\end{verbatim}


\subsubsection{Generating service}

With evegenie installed, the generated JSON file from the above step
is processed to create the stub REST service definitions.
