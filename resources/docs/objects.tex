\documentclass[9pt,twocolumn,twoside]{styles/osajnl} 
\usepackage{fancyvrb}
\journal{cm}  

%\usepackage{xcolor}
%usepackage{fullpage}
%\usepackage{fancyvrb}
%\renewcommand{\familydefault}{\sfdefault}
%\usepackage[scaled=0.92]{helvet}
%\usepackage[helvet]{sfmath}
%\everymath={\sf}
%\parindent 0pt

\newcommand{\tightlist}{}

\title{Cloudmesh REST Interface for Virtual Clusters} 

\author[1,*]{Gregor von Laszewski} 
\author[1]{Fugang Wang}
\author[1]{Badi Abdhul-Wahid}

\affil[1]{School of Informatics and Computing, Bloomington, IN 47408, U.S.A.} 
\affil[*]{Corresponding authors: laszewski@gmal.com} 

\dates{Draft v0.0.1, \today} 

\ociscodes{CLoudmesh, REST, NIST} 

\doi{\url{https://github.com/cloudmesh/rest/tree/master/resources/docs}} 


\begin{abstract}

This document summarizes a number of objects that are instrumental for
the interaction with Clouds, Containers, and HPC systems to manage
virtual clusters. 
TBD

\end{abstract}

\begin{document}


\flushbottom % Makes all text pages the same height

\maketitle % Print the title and abstract box

\tableofcontents % Print the contents section
\maketitle



\section{Contributing}

We invite you to contribute to this paper and its discussion to
improve it. Improvements can be done with pull requests. We suggest
you do {\em small} individual changes to a single section and object
rather than large changes as this allows us to integrate the changes
individually and comment on your contribution via github.

Once contributed we will appropriately acknoledge you either as
contributor or author. Please discuss with us how we best acknowledge
you.

\section{Using the Cloudmesh REST Service} 

Components are written as YAML markup in files in the
\verb+resources/samples+ directory.

For example:

\VerbatimInput{specification/profile.yml}

\subsection{Element Definition}

Each resource should have a \verb+description+ entry to act as
documentation. The documentation should be formated as
reStructuredText. For example:

\subsection{Yaml}

\begin{Verbatim}
entry = yaml.read('''
profile:
  description: |
    A user profile that specifies general information 
    about the user
  email: laszewski@gmail.com, required
  firtsname: Gregor, required
  lastname: von Laszewski, required
  height: 180
'''}
\end{Verbatim}

\subsection{Cerberus}

\begin{Verbatim}
schema = {
'profile': {
  'description': {'type': 'string'}
  'email':       {'type': 'string', 'required': True}
  'firtsname':   {'type': 'string', 'required': True}
  'lastname':    {'type': 'string', 'required': True}
  'height':      {'type': 'float'}
}
\end{Verbatim}

\section{Mongoengine}

\begin{Verbatim}
class profile(Document):
    description = StringField()
    email = EmailField(required=True)
    firstname = StringField(required=True)
    lastname = StringField(required=True)
    height = FloatField(max_length=50)
\end{Verbatim}

\section{Cloudmesh Notation}

\begin{Verbatim}
profile:
    description: string
    email: email, required
    firstname: string, required
    lastname: string, required
    height: flat, max=10
\end{Verbatim}

\begin{Verbatim}
proposed command

cms schema FILENAME --format=mongo -o OUTPUT
cms schema FILENAME --format=cerberus -o OUTPUT
cms schema FILENAME --format=yaml -o OUTPUT

  reads FILENAME in cloudmesh notation and returns format


cms schema FILENAME --input=evegenie -o OUTPUT
   reads eavegene example and create settings for eve
\end{Verbatim}


\subsection{Defining Elements for the REST Service}

To manage a large number of elements defined in our REST service
easily, we manage them trhough definitions in yaml files. To generate
the appropriate settings file for the rest service, we can use teh
following command:

\begin{verbatim}
cms admin elements <directory> <out.json>
\end{verbatim}

where

\begin{itemize}
\item \verb+<directory>+: directory where the YAML definitions reside
\item \verb+<out.json>+: path to the combined definition
\end{itemize}

For example, to generate a file called all.json that integrates all
yml objects defined in the directory \verb+resources/samples+ you can
use the following command:

\begin{verbatim}
cms elements resources/samples all.json
\end{verbatim}

\subsection{DOIT}


cms schema spec2tex resources/specification resources/tex

\subsection{Generating service}

With evegenie installed, the generated JSON file from the above step
is processed to create the stub REST service definitions.


\section{Network}

We are looking for volunteers to contribute here.

\section{NEW}

\subsection{azure-size-simple}

\VerbatimInput{specification/azure-size-simple.yml}
The size description of an azure vm


\subsection{azure-vm-simple}

\VerbatimInput{specification/azure-vm-simple.yml}
An Azure virtual machine


\subsection{batchjob-simple}

\VerbatimInput{specification/batchjob-simple.yml}
\input{specification/batchjob.tex}

\subsection{cluster-simple}

\VerbatimInput{specification/cluster-simple.yml}
The cluster object has name, label, endpoint and provider. The \textit{endpoint}
defines.... The \textit{provider} defines the nature of the cluster,
e.g., it's a virtual cluster on an openstack cloud, or from AWS, or a bare-metal
cluster.


\subsection{compute resource-simple}

\VerbatimInput{specification/compute_resource-simple.yml}
\textbf{compute\_resource} object has attribute \textit{endpoint} which
specifies ... The \textit{kind} could be \textit{baremetal} or \textit{VC}.


\subsection{computer-simple}

\VerbatimInput{specification/computer-simple.yml}
This defines a \textbf{computer} object. A computer has name, label,
IP address. It also listed the relevant specs such as memory, disk size, etc.

\subsection{container-simple}

\VerbatimInput{specification/container-simple.yml}
This defines \textbf{container} object.


\subsection{database-simple}

\VerbatimInput{specification/database-simple.yml}
A \textbf{database} could have a name, an \textit{endpoint} (e.g., host:port),
and protocol used (e.g., SQL, mongo, etc.).


\subsection{default-simple}

\VerbatimInput{specification/default-simple.yml}
\input{specification/default.tex}

\subsection{deployment-simple}

\VerbatimInput{specification/deployment-simple.yml}
A \textbf{deployment} consists of the resource \- \textit{cluster},
the location \- \textit{provider}, e.g., AWS, OpenStack, etc., and
software \textit{stack} to be deployed (e.g., hadoop, spark).


\subsection{file-simple}

\VerbatimInput{specification/file-simple.yml}
The \textbf{file} object has \textit{name}, \textit{endpoint} (location), \textit{size}
in GB, MB, Byte, \textit{checksum} for integrity check, and last \textit{accessed} timestamp.


\subsection{file alias-simple}

\VerbatimInput{specification/file_alias-simple.yml}
A file could have one alias or even multiple ones.

\subsection{hadoop-simple}

\VerbatimInput{specification/hadoop-simple.yml}
A \textbf{hadoop} definition defines which \textit{deployer} to be used,
the \textit{parameters} of the deployment, and the system packages as
\textit{requires}. For each requirement, it could have attributes such
as the library origin, version, etc.


\subsection{key-simple}

\VerbatimInput{specification/key-simple.yml}
\input{specification/key.tex}

\subsection{kubernetes-simple}

\VerbatimInput{specification/kubernetes-simple.yml}
\input{specification/kubernetes.tex}

\subsection{mapreduce-simple}

\VerbatimInput{specification/mapreduce-simple.yml}
This defines a \textbf{mapreduce} deployment with its layered components.

\subsection{mesos-simple}

\VerbatimInput{specification/mesos-simple.yml}
\input{specification/mesos.tex}

\subsection{microservice-simple}

\VerbatimInput{specification/microservice-simple.yml}
A system could be composed of from various microservices, and this defines
each of them.

\subsection{node-simple}

A node is composed of multiple components:Metadata such as the
{\em name}  or {\em owner} .  Physical properties such as {\em cores}  or
{\em memory} .  Configuration guidance such as {\em create\_external\_ip} ,
{\em security\_groups} , or {\em users} . The metadata is associated with the
node on the provider end (if supported) as well as in the database.
Certain parts of the metadata (such as {\em owner} ) can be used to
implement access control.  Physical properties are relevant for the
initial allocation of the node.  Other configuration parameters
control and further provisioning.  In the above, after allocation, the
node is configured with a user called {\em hello}  who is part of the
{\em wheel}  group whose account can be accessed with several SSH
identities whose public keys are provided (in
{\em authorized\_keys} ). Additionally, three ssh keys are generated on
the node for the {\em hello}  user.  The first uses the {\em
      ed25519}    
cryptographic method with a password read in from a GPG-encrypted file
on the Command and Control node.  The second is a 4098-bit RSA key
also password-protected from the GPG-encrypted file.  The third key is
copied to the remote node from an encrypted file on the Command and
Control node.  This definition also provides a security group to
control access to the node from the wide-area-network.  In this case
all ingress and egress TCP and UDP traffic is allowed provided they
are to ports 22 (SSH), 443 (SSL), and 80 and 8080 (web). "


\VerbatimInput{specification/node-simple.yml}

\subsection{openstack flavor-simple}

\VerbatimInput{specification/openstack_flavor-simple.yml}
OpenStack flavor object


\subsection{openstack image-simple}

\VerbatimInput{specification/openstack_image-simple.yml}
OpenStack image object


\subsection{openstack vm-simple}

\VerbatimInput{specification/openstack_vm-simple.yml}
\input{specification/openstack_vm.tex}

\subsection{profile-simple}

\VerbatimInput{specification/profile-simple.yml}
This object defines a user profile.

\subsection{replica-simple}

\VerbatimInput{specification/replica-simple.yml}
The \textbf{replica} defines a replica for a file.

See also \textbf{file} and \textbf{file\_alias}.


\subsection{reservation-simple}

\VerbatimInput{specification/reservation-simple.yml}
\input{specification/reservation.tex}

\subsection{user-simple}

\VerbatimInput{specification/user-simple.yml}
This defines a user object. Besides the profile information, this may have
other necessary information, e.g., passwrod hash, roles, etc., to facilitate
the Authentication/Authorization.

\subsection{var-simple}

\VerbatimInput{specification/var-simple.yml}
\input{specification/var.tex}

\subsection{virtual cluster-simple}

\VerbatimInput{specification/virtual_cluster-simple.yml}
virtual_cluster:
  endpoint: http://.../cluster/
  name: data
  nodes:
  - virtual_compute_node


\subsection{virtual compute node-simple}

\VerbatimInput{specification/virtual_compute_node-simple.yml}
virtual\_compute\_node:
  endpoint: http://.../cluster/
  flavor: TBD
  image: Ubuntu-16.04
  ip:
  \- TBD
  metadata:
    experiment: exp-001
  name: data
  status: TBD


\subsection{virtual directory-simple}

\VerbatimInput{specification/virtual_directory-simple.yml}
virtual_directory:
  collection:
  - report.dat
  - file2
  endpoint: http://.../data/
  name: data
  protocol: http


\subsection{virtual machine-simple}

\VerbatimInput{specification/virtual_machine-simple.yml}
virtual_machine:
  endpoint: http://.../vm/
  flavor: TBD
  image: Ubuntu-16.04
  ip:
  - TBD
  metadata:
    experiment: exp-001
  name: vm1
  status: TBD


\section{Schema Command}

\VerbatimInput{schema-man.tex}
\newpage

\appendix

{|bf README.rst}
\section{Cloudmesh Rest}\label{s:cloudmesh-rest}

Cloudmesh Resst is a refernce implementattion for the NBDRA. It
allows to define automatically a REST service based on the objects
specified by the NBDRA document. In collaboration with other cloudmesh
components it allows easy interaction with hybrid clouds and the
creation of user managed big data services. 

\subsection{Prerequistis}\label{prerequistis}

The preriquisits for Cloudmesh REST are Python 2.7.13 or 3.6.1
it can easily be installed on a variety of systems (at this time we
have only tried ubuntu greater 16.04 and OSX Sierra. However, it would
naturally be possible to also port it to Windows. The instalation
instruction in this document are not complete and we recommend to
refer to the cloudmesh manuals which are under development. The goal
will be to make the instalation (after your system is set up for
developing python) as simple as 

\begin{verbatim}
    pip install cloudmesh.rest
\end{verbatim}


\subsection{REST Service}\label{cm-rest}

With the cloudmesh REST framework it is easy to create REST services
while defining the resources via example json objects. This is achieved
while leveraging the python eve \cite{www-eve} and a modified version of python
evengine \cite{www-cloudmesh-eveengine}. 

A valid json resource specification looks like this:

\begin{verbatim}
{
  "profile": {
    "description": "The Profile of a user",
    "email": "laszewski@gmail.com",
    "firstname": "Gregor",
    "lastname": "von Laszewski",
    "username": "gregor"
  }
}
\end{verbatim}

here we define an object called profile, that contains a number of
attributes and values. The type of the values are automatically
determined. All json specifications are contained in a directory and
can easily be converted into a valid schema for the eve rest service
by executing the commands

\begin{verbatim}
cms schema cat . all.json
cms schema convert all.json
\end{verbatim}

This will create a the configuration \verb|all.settings.py| that can
be used to start an eve service

Once the schema has defined, cloudmesh specifies defaults for managing
a sample data base that is coupled with the REST service. We use
mongodb which could be placed on a sharded mongo service. 

\subsection{Limitations}

The current implementation is a demonstration and showcases that it is
easy to generate a fully functioning REST service based on the
specifications provided in this document. However, it is expected that
scalability, distribution of services, and other advanced options
need to be addrassed based on application requirements.


classes lessons rest.rst
\section{REST with Eve}\label{rest-with-eve}

\subsection{Overview of REST}\label{overview-of-rest}

REST stands for REpresentational State Transfer. REST is an architecture
style for designing networked applications. It is based on stateless,
client-server, cacheable communications protocol. Although not based on
http, in most cases, the HTTP protocol is used. In contrast to what some
others write or say, REST is not a \emph{standard}.

RESTful applications use HTTP requests to:

\begin{itemize}
\tightlist
\item
  post data: while creating and/or updating it,
\item
  read data: while making queries, and
\item
  delete data.
\end{itemize}

Hence REST uses HTTP for the four CRUD operations:

\begin{itemize}
\tightlist
\item
  Create
\item
  Read
\item
  Update
\item
  Delete
\end{itemize}

As part of the HTTP protocol we have methods such as GET, PUT, POST, and
DELETE. These methods can than be used to implement a REST service. As
REST introduces collections and items we need to implement the CRUD
functions for them. The semantics is explained in the Table
illustrationg how to implement them with HTTP methods.

Source:
\url{https://en.wikipedia.org/wiki/Representational_state_transfer}

\subsection{REST and eve}\label{rest-and-eve}

Now that we have outlined the basic functionality that we need, we lke
to introduce you to Eve that makes this process rather trivial. We will
provide you with an implementation example that showcases that we can
create REST services without writing a single line of code. The code for
this is located at \url{https://github.com/cloudmesh/rest}

This code will have a master branch but will also have a dev branch in
which we will add gradually more objects. Objects in the dev branch will
include:

\begin{itemize}
\tightlist
\item
  virtual directories
\item
  virtual clusters
\item
  job sequences
\item
  inventories
\end{itemize}

;You may want to check our active development work in the dev branch.
However for the purpose of this class the master branch will be
sufficient.

\subsubsection{Installation}\label{installation}

First we havt to install mongodb. The instalation will depend on your
operating system. For the use of the rest service it is not important to
integrate mongodb into the system upon reboot, which is focus of many
online documents. However, for us it is better if we can start and stop
the services explicitly for now.

On ubuntu, you need to do the following steps:

\begin{verbatim}
TO BE CONTRIBUTED BY THE STUDENTS OF THE CLASS as homework
\end{verbatim}

On windows 10, you need to do the following steps:

\begin{verbatim}
TO BE CONTRIBUTED BY THE STUDENTS OF THE CLASS as homework, if you
elect Windows 10. YOu could be using the online documentation
provided by starting it on Windows, or rinning it in a docker container.
\end{verbatim}

On OSX you can use homebrew and install it with:

\begin{verbatim}
brew update
brew install mongodb
\end{verbatim}

\begin{description}
\item[In future we may want to add ssl authentication in which case you
may]
need to install it as follows:
\end{description}

brew install mongodb --with-openssl

\subsubsection{Starting the service}\label{starting-the-service}

We have provided a convenient Makefile that currently only works for
OSX. It will be easy for you to adapt it to Linux. Certainly you can
look at the targes in the makefile and replicate them one by one.
Improtaht targest are deploy and test.

When using the makefile you can start the services with:

\begin{verbatim}
make deploy
\end{verbatim}

IT will start two terminals. IN one you will see the mongo service, in
the other you will see the eve service. The eve service will take a file
called sample.settings.py that is base on sample.json for the start of
the eve service. The mongo servide is configured in suc a wahy that it
only accepts incimming connections from the local host which will be
suffiicent fpr our case. The mongo data is written into the
\$USER/.cloudmesh directory, so make sure it exists.

To test the services you can say:

\begin{verbatim}
make test
\end{verbatim}

YOu will se a number of json text been written to the screen.

\subsection{Creating your own objects}\label{creating-your-own-objects}

The example demonstrated how easy it is to create a mongodb and an eve
rest service. Now lets use this example to creat your own. FOr this we
have modified a tool called evegenie to install it onto your system.

The original documentation for evegenie is located at:

\begin{itemize}
\tightlist
\item
  \url{http://evegenie.readthedocs.io/en/latest/}
\end{itemize}

However, we have improved evegenie while providing a commandline tool
based on it. The improved code is located at:

\begin{itemize}
\tightlist
\item
  \url{https://github.com/cloudmesh/evegenie}
\end{itemize}

You clone it and install on your system as follows:

\begin{verbatim}
cd ~/github
git clone https://github.com/cloudmesh/evegenie
cd evegenie
python setup.py install
pip install .
\end{verbatim}

This shoudl install in your system evegenie. YOu can verify this by
typing:

\begin{verbatim}
which evegenie
\end{verbatim}

If you see the path evegenie is installed. With evegenie installed its
usaage is simple:

\begin{verbatim}
$ evegenie

Usage:
    evegenie --help
    evegenie FILENAME
\end{verbatim}

It takes a json file as input and writes out a settings file for the use
in eve. Lets assume the file is called sample.json, than the settings
file will be called sample.settings.py. Having the evegenie programm
will allow us to generate the settings files easily. You can include
them into your project and leverage the Makefile targets to start the
services in your project. In case you generate new objects, make sure
you rerun evegenie, kill all previous windows in whcih you run eve and
mongo and restart. In case of changes to objects that you have designed
and run previously, you need to also delete the mongod database.

\subsection{Towards cmd5 extensions to manage eve and
mongo}\label{towards-cmd5-extensions-to-manage-eve-and-mongo}

Naturally it is of advantage to have in cms administration commands to
manage mongo and eve from cmd instead of targets in the Makefile. Hence,
we \textbf{propose} that the class develops such an extension. We will
create in the repository the extension called admin and hobe that
students through collaborative work and pull requests complete such an
admin command.

The proposed command is located at:

\begin{itemize}
\tightlist
\item
  \url{https://github.com/cloudmesh/rest/blob/master/cloudmesh/ext/command/admin.py}
\end{itemize}

It will be up to the class to implement such a command. Please
coordinate with each other.

The implementation based on what we provided in the Make file seems
straight forward. A great extensinion is to load the objects definitions
or eve e.g. settings.py not from the class, but forma place in
.cloudmesh. I propose to place the file at:

\begin{verbatim}
.cloudmesh/db/settings.py
\end{verbatim}

the location of this file is used whne the Service class is initialized
with None. Prior to starting the service the file needs to be copied
there. This could be achived with a set commad.

classes lesson python cmd5.rst
\section{CMD5}\label{cmd5}

CMD is a very useful package in python to create command line shells.
However it does not allow the dynamic integration of newly defined
commands. Furthermore, addition to cmd need to be done within the same
source tree. To simplify developping commands by a number of people and
to have a dynamic plugin mechnism, we developed cmd5. It is a rewrite on
our ealier effords in cloudmesh and cmd3.

\subsection{Resources}\label{resources}

The source code for cmd5 is located in github:

\begin{itemize}
\tightlist
\item
  \url{https://github.com/cloudmesh/cmd5}
\end{itemize}

Installation from source -----------------------

We recommend that you use a virtualenv either with virtualenv or pyenv.
This can be either achieved vor virtualenv with:

\begin{verbatim}
virtualenv ~/ENV2
\end{verbatim}

or for pyenv, with:

\begin{verbatim}
pyenev virtualenv 2.7.13 ENV2
\end{verbatim}

Now you need to get two source directories. We assume yo place them in
\textasciitilde{}/github:

\begin{verbatim}
mkdir ~/github
cd ~/github

git clone https://github.com/cloudmesh/common.git
git clone https://github.com/cloudmesh/cmd5.git
git clone https://github.com/cloudmesh/extbar.git

cd ~/github/common
python setup.py install
pip install .

cd ~/github/cmd5
python setup.py install
pip install .

cd ~/github/extbar
python setup.py install
pip install .
\end{verbatim}

The cmd5 repository contains the shell, while the extbar directory
contains the sample to add the dynamic commands foo and bar.

\subsection{Execution}\label{execution}

To run the shell you can activate it with the cms command. cms stands
for cloudmesh shell:

\begin{verbatim}
(ENV2) $ cms
\end{verbatim}

It will print the banner and enter the shell:

\begin{verbatim}
+-------------------------------------------------------+
|   ____ _                 _                     _      |
|  / ___| | ___  _   _  __| |_ __ ___   ___  ___| |__   |
| | |   | |/ _ \| | | |/ _` | '_ ` _ \ / _ \/ __| '_ \  |
| | |___| | (_) | |_| | (_| | | | | | |  __/\__ \ | | | |
|  \____|_|\___/ \__,_|\__,_|_| |_| |_|\___||___/_| |_| |
+-------------------------------------------------------+
|                  Cloudmesh CMD5 Shell                 |
+-------------------------------------------------------+

cms>
\end{verbatim}

To see the list of commands you can say

\begin{quote}
cms\textgreater{} help
\end{quote}

To see the manula page for a specific command, please use:

\begin{verbatim}
help COMMANDNAME
\end{verbatim}

\subsection{Create your own Extension}\label{create-your-own-extension}

One of the most important features of CMD5 is its ability to extend it
with new commands. This is done via packaged name spaces. This is
defined in the setup.py file of your enhancement. The best way to create
an enhancement is to take a look at the code in

\begin{itemize}
\tightlist
\item
  \url{https://github.com/cloudmesh/extbar.git}
\end{itemize}

Simply copy the code and modify the bar and foo commands to fit yor
needs.

\begin{description}
\item[make sure you are not copying the .git directory. Thus we]
recommend that you copy it explicitly file by file or directory by
directory
\end{description}

It is important that all objects are defined in the command itself and
that no global variables be use in order to allow each shell command to
stand alone. Naturally you should develop API libraries outside of the
cloudmesh shell command and reuse them in order to keep the command code
as small as possible. We place the command in:

\begin{verbatim}
cloudmsesh/ext/command/COMMANDNAME.py
\end{verbatim}

An example for the bar command is presented at:

\begin{itemize}
\tightlist
\item
  \url{https://github.com/cloudmesh/extbar/blob/master/cloudmesh/ext/command/bar.py}
\end{itemize}

It shows how simple the command definition is (bar.py):

\begin{verbatim}
from __future__ import print_function
from cloudmesh.shell.command import command
from cloudmesh.shell.command import PluginCommand

class BarCommand(PluginCommand):

    @command
    def do_bar(self, args, arguments):
        """
        ::
          Usage:
                command -f FILE
                command FILE
                command list
          This command does some useful things.
          Arguments:
              FILE   a file name
          Options:
              -f      specify the file
        """
        print(arguments)
\end{verbatim}

An important difference to other CMD solutions is that our commands can
leverage (besides the standrad definition), docopts as a way to define
the manual page. This allows us to use arguments as dict and use simple
if conditions to interpret the command. Using docopts has the advantage
that contributors are forced to think about the command and its options
and document them from the start. Previously we used not to use docopts
and argparse was used. However we noticed that for some contributions
the lead to commands that were either not properly documented or the
developers delivered ambiguous commands that resulted in confusion and
wrong ussage by the users. Hence, we do recommend that you use docopts.

The transformation is enabled by the @command decorator that takes also
the manual page and creates a proper help message for the shell
automatically. Thus there is no need to introduce a sepaarte help method
as would normally be needed in CMD.

\subsection{Excersise}\label{excersise}

\begin{description}
\item[CMD5.1:]
Install cmd5 on your computer.
\item[CMD5.2:]
Write a new command with your firstname as the command name.
\item[CMD5.3:]
Write a new command and experiment with docopt syntax and argument
interpretation of the dict with if conditions.
\item[CMD5.4:]
If you have useful extensions that you like us to add by default, please
work with us.
\end{description}


\end{document}



