\section{Cloudmesh Rest}\label{cloudmesh-rest}

\subsection{Prerequistis}\label{prerequistis}

\begin{itemize}
\tightlist
\item
  mongo instalation
\item
  eve instalation
\item
  cloudmesh cmd5
\item
  cloudmesh rest
\end{itemize}

\subsubsection{Install Mongo on OSX}\label{install-mongo-on-osx}

\begin{verbatim}
brew update
brew install mongodb

# brew install mongodb --with-openssl
\end{verbatim}

\subsubsection{Install Mongo on OSX}\label{install-mongo-on-osx-1}

ASSIGNMET TO STUDENTS, PROVIDE PULL REQUEST WITH INSTRUCTIONS

\subsection{Introduction}\label{introduction}

With the cloudmesh REST framework it is easy to create REST services
while defining the resources in the service easily with examples. The
service is based on eve and the examples are defined in yml to be
converted to json and from json with evegenie into a valid eve settings
file.

Thus oyou can eother wite your examples in yaml or in json. The
resources are individually specified in a directory. The directory can
contain multiple resource files. We recomment that for each resource you
define your own file. Conversion of the specifications can be achieved
with the schema command.

\subsection{Yaml Specification}\label{yaml-specification}

Let us first introduce you to a yaml specification. Let us assume that
your yaml file is called profile.yaml and located in a directory called
`example`:

\begin{verbatim}
profile:
  description: The Profile of a user
  email: laszewski@gmail.com
  firstname: Gregor
  lastname: von Laszewski
  username: gregor
\end{verbatim}

As eve takes json objects as default we need to convert it first to
json. This is achieved wih:

\begin{verbatim}
cd example
cms schema convert profile.yml profile.json
\end{verbatim}

This will provide the json file profile.json as Listed in the next
section

\subsection{Json Specification}\label{json-specification}

A valid json resource specification looks like this:

\begin{verbatim}
{
  "profile": {
    "description": "The Profile of a user",
    "email": "laszewski@gmail.com",
    "firstname": "Gregor",
    "lastname": "von Laszewski",
    "username": "gregor"
  }
}
\end{verbatim}

\subsection{Conversion to Eve
Settings}\label{conversion-to-eve-settings}

The json files in the \textasciitilde{}/sample directory need now to be
converted to a valid eve schema. This is achieved with tow commands.
First, we must concatenate all json specified resource examples into a
single json file. We do this with:

\begin{verbatim}
cms schema cat . all.json
\end{verbatim}

As we assume you are in the samples directory, we use a . for the
current location of the directory that containes the samples. Next, we
need to convert it to the settings file. THis can be achieved with the
convert program when you specify a json file:

\begin{verbatim}
cms schema convert all.json
\end{verbatim}

THe result will be a eve configuration file that you can use to start an
eve service. The file is called all.settings.py

\subsubsection{Managing Mongo}\label{managing-mongo}

Next you need to start the mongo service with

\begin{verbatim}
cms admin mongo start
\end{verbatim}

You can look at the status and information about the service with :

\begin{verbatim}
cms admin mongo info
cms admin mongo status
\end{verbatim}

If you need to stop the service you can use:

\begin{verbatim}
cms admin mongo stop
\end{verbatim}

\subsubsection{Manageing Eve}\label{manageing-eve}

Now it is time to start the REST service. THis is done in a separate
window with the following commands:

\begin{verbatim}
cms admin settings all.settings.json
cms admin rest start
\end{verbatim}

The first command coppies the settings file to

\begin{quote}
\textasciitilde{}/cloudmesh/eve/settings.py
\end{quote}

This file is than used by the start action to start the eve service.
Please make sure that you execute this command in a separate window, as
for debugging purposses you will be able to monitor this way
interactions with this service

Testing - OLD \^{}\^{}\^{}\^{}\^{}\^{}\^{} :

\begin{verbatim}
make setup    # install mongo and eve
make install  # installs the code and integrates it into cmd5
make deploy
make test
\end{verbatim}
