\section{Cloudmesh Rest}\label{s:cloudmesh-rest}

Cloudmesh Resst is a refernce implementattion for the NBDRA. It
allows to define automatically a REST service based on the objects
specified by the NBDRA document. In collaboration with other cloudmesh
components it allows easy interaction with hybrid clouds and the
creation of user managed big data services. 

\subsection{Prerequistis}\label{prerequistis}

The preriquisits for Cloudmesh REST are Python 2.7.13 or 3.6.1
it can easily be installed on a variety of systems (at this time we
have only tried ubuntu greater 16.04 and OSX Sierra. However, it would
naturally be possible to also port it to Windows. The instalation
instruction in this document are not complete and we recommend to
refer to the cloudmesh manuals which are under development. The goal
will be to make the instalation (after your system is set up for
developing python) as simple as 

\begin{verbatim}
    pip install cloudmesh.rest
\end{verbatim}


\subsection{REST Service}\label{cm-rest}

With the cloudmesh REST framework it is easy to create REST services
while defining the resources via example json objects. This is achieved
while leveraging the python eve \cite{www-eve} and a modified version of python
evengine \cite{www-cloudmesh-eveengine}. 

A valid json resource specification looks like this:

\begin{verbatim}
{
  "profile": {
    "description": "The Profile of a user",
    "email": "laszewski@gmail.com",
    "firstname": "Gregor",
    "lastname": "von Laszewski",
    "username": "gregor"
  }
}
\end{verbatim}

here we define an object called profile, that contains a number of
attributes and values. The type of the values are automatically
determined. All json specifications are contained in a directory and
can easily be converted into a valid schema for the eve rest service
by executing the commands

\begin{verbatim}
cms schema cat . all.json
cms schema convert all.json
\end{verbatim}

This will create a the configuration \verb|all.settings.py| that can
be used to start an eve service

Once the schema has defined, cloudmesh specifies defaults for managing
a sample data base that is coupled with the REST service. We use
mongodb which could be placed on a sharded mongo service. 

\subsection{Limitations}

The current implementation is a demonstration and showcases that it is
easy to generate a fully functioning REST service based on the
specifications provided in this document. However, it is expected that
scalability, distribution of services, and other advanced options
need to be addrassed based on application requirements.

