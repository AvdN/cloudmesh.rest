\section*{Executive Summary}

The NIST Big Data Interoperability Framework: Volume 8 document
\cite{nist-vol-6} was prepared by the NIST Big Data Public Working
Group (NBD-PWG) Interface Subgroup to identify interfaces in support
of the NIST Big Data Reference Architecture (NBDRA) The interafces
contain two differnt aspects:

\begin{itemize}

\item the definition of resources that are part of the NBDRA. These
  resources are formulated in Json format and can be integrated into a
  REST framework or an object based framework easily.

\item the definition of simple interface use cases that allow us to
  illustrate the usefulness of the resources defined.

\end{itemize} 

We categorized the resources in groups that are identified by the
NBDRA set forward in Volume 6. While Volume 3 provides {\it
  application} oriented high level use cases the usecases defined in
this document are subsets of them and focus on {\it interface} use
cases. The interface use cases are not meant to be complete examples,
but showcase why the resource has been defined. Hence, the interfaces
use cases are, of course, only representative, and do not represent
the entire spectrum of Big Data usage. All of the interfaces were
openly discussed in the working group. Additions are welcome and we
like to dicuss your contributions in the group.

The NIST Big Data Interoperability Framework consists of nine
volumes, each of which addresses a specific key topic, resulting from
the work of the NBD-PWG. The eight volumes are:

\begin{itemize}
\item Volume 1, Definitions
\item Volume 2, Taxonomies
\item Volume 3, Use Cases and General Requirements
\item Volume 4, Security and Privacy
\item Volume 5, Architectures White Paper Survey
\item Volume 6, Reference Architecture
\item Volume 7, Standards Roadmap
\item Volume 8, Interfaces
\item Volume 9, Big Data Adoption and Modernization
\end{itemize}

The NIST Big Data Interoperability Framework will be released in three
versions, which correspond to the three development stages of the
NBD-PWG work. The three stages aim to achieve the following with
respect to the NIST Big Data Reference Architecture (NBDRA).

\begin{itemize}
\item Stage 1: Identify the high-level Big Data reference architecture
  key components, which are technology-, infrastructure-, and
  vendor-agnostic.
\item Stage 2: Define general interfaces between the NBDRA components.
\item Stage 3: Validate the NBDRA by building Big Data general
  applications through the general interfaces.
\end{itemize}

This document is targeting Stage 2 of the NBDRA. Coordination of the
group is conducted on its Web page \cite{www-nbdwg}. 